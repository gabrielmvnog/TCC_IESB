Esta seção tem como objetivo apresentar os conhecimentos necessários para o claro entendimento deste trabalho.

\section{Aprendizado de Máquina}
Na década de 1970, a disseminação de técnicas no ramo da inteligência artificial (IA), tiveram um avanço na solução de problemas reais. A fim de solucionar tais problemas, muitas vezes eram necessários especialistas de um dado domínio para fornecer conhecimento, para com isso, posteriormente ser estruturado computacionalmente e ser desenvolvido a solução. Tal conhecimento era estruturado por um conjunto de regras lógicas em programas de computador, que ficaram conhecidos como Sistemas Especialistas (SE) ou Sistemas Baseados em Conhecimento (SBC). Porém, quando depende-se de um ser humano, alguns fatores podem ocorrer no processo de aquisição do conhecimento, que é a subjetividade ou até mesmo a dificuldade de transmitir o conhecimento por parte do especialista, podendo assim tornar o processo ineficaz por falta de informações.

Com o passar do tempo e o avanço da tecnologia, a quantidade de dados cresceu exponencialmente, acarretando no surgimento de problemas mais complexos a serem resolvidos computacionalmente. Com isso, surge uma maior necessidade de se ter ferramentas computacionais mais independentes com a finalidade de reduzir a interferência humana e a dependência de especialistas. Tais ferramentas, devem ser capazes de gerar uma saída, baseado em experiências e dados refentes ao passado, ou hipóteses, a fim de solucionar um dado problema. Um exemplo disso é a sugestão de produtos que um cliente provavelmente possa se interessar. Essa sugestão pode ser feita através de produtos previamente comprados pelo cliente, gerando hipóteses de outros produtos através de algum padrão encontrado entre outros produtos, podendo ser alguns deles: marca, categoria ou até mesmo faixa de preço. Esse processo de indução da hipótese pela ferramente computacional, é chamado de Aprendizado de Máquina (AM).

\subsection{Aprendizado de Máquina aplicado à Biologia}
No ramo biologia tem se tornado frequente ver aplicações com o intuito de bla bla bla

\subsection{Redes Neurais Artificiais}
* Falar sobre história de rede neural? Aplicações atualmente?

* Estrutura de uma rede neural.

* Falar de ML aplicado à biologia e o que pode ser feito.

\subsection{Técnicas de validação}
* Técnicas de validação

\subsection{Métricas}
* Métricas

\subsection{Algoritmos}
* Falar dos algoritmos que iremos utilizar.

Rede Neural RNA1
RNA2