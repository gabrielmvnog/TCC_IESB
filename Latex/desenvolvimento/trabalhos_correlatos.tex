\section{Trabalhos Correlatos}

Em \citeauthoronline{tanaka2013adaptaccao} (\citeyear{tanaka2013adaptaccao}), é utilizado uma adaptação do método \textit{Binary Relevance} com árvores de decisão para classificação multirrotular de proteínas. Por conta da base de dados ser estruturada de forma hierárquica, a classificação foi feita em cada um dos 4 níveis de classificação da base, chegando a ter mais de 95\% de acurácia em alguns rótulos.

Segundo \citeauthoronline{resende2018combinaccao} (\citeyear{resende2018combinaccao}) propõe o uso de redes complexas com algoritmos de classificação para aplicação em problemas de classificação multirrótulo, utilizando-se de conceitos em teoria dos grafos para levar a consideração a topologia dos dados e, assim, melhorar a inferência dos rótulos. Em uma das bases testadas, foi possível observar um ganho de aproximadamente 20\% em acurácia com o uso desta técnica.

Em \citeauthoronline{ururahy2018classificaccao} (\citeyear{ururahy2018classificaccao}) é empregado o uso de classificação multirrótulo para classificar os possíveis gêneros de um filme a partir do sumário. No trabalho, foram escolhidos os 500 termos mais frequentes dos resumos para a construção do classificador, usando, para isso, o método \textit{Binary Relevance} e o algoritmo \texttt{Naive Bayes}. A acurácia máxima, média e mínima dentre os rótulos utilizados foram de 76\%, 66\% e 62\%, respectivamente. 

\citeauthoronline{coelho2018prediccao} (\citeyear{coelho2018prediccao}) faz o uso de máquinas de vetor de suporte e RNAa para predizer regiões promotoras. Os testes foram efetuados com dados da bactéria \textit{Bacillus subtilis}, uma Gram-positiva, já que “grande parte dos trabalhos sobre promotores bacterianos estão restritos a bactérias Gram-negativas”, além de existir “poucos dados de Gram-positivas na literatura” \cite{coelho2018prediccao}. O estudo concluiu que os resultados observados estão de acordo com a literatura, chegando a obter 98.57\% e 97.69\% de acurácia, no caso de duas redes neurais com configurações similares, e 93.20\% e 95.63\% de acurácia, no caso de máquinas de vetor de suporte com diferentes \texttt{kernels}.

 Em \citeauthoronline{zhang2014review} (\citeyear{zhang2014review}) é feito um estudo do estado da arte em aprendizagem multirrótulo, abordando os fundamentos, um detalhamento técnico entre 8 algoritmos e faz um breve comparativo com a aprendizagem supervisionada. O autor expõe a importância e a relevância de em certos momentos, termos uma instância associada à mais de um rótulo. Os algoritmos citados e expostos pelo autor, são: \textit{Binary Relevance}, \textit{Classifier Chains}, \textit{Calibrated Label Ranking}, \textit{Random k-Labelsets}, a adaptação do \textit{k-Nearest Neighbor} (kNN) para classificação multirrótulo (ML-kNN), \textit{Multi-Label Decision Tree} (ML-DT), \textit{Ranking Support Vector Machine} (Rank-SVM) e \textit{Collective Multi-Label Classifier} (CML).

\citeauthoronline{gibaja2014} (\citeyear{gibaja2014}) fazem uma compilação sobre o estado da arte em classificação multirrótulo. A pesquisa apresenta um olhar da área sobre diversas perspectivas, tais como: aplicações, estudos em andamento, a CMR na literatura, os principais algoritmos existentes, métricas de avaliação, entre outros.

\citeauthoronline{liu2015multi} (\citeyear{liu2015multi}) apresentam um estudo comparativo entre diferentes algoritmos de classificação multirrótulo aplicado à textos de microblogs, utilizando, para isso, duas bases de dados distintas: HR e AI
\footnote{As bases de dados HR e AI são compostas de comentários sobre dois incidentes que aconteceram na China, os quais ganharam atenção nacional na época em que aconteceram. Mais informações em \loccit{liu2015multi}}. 
O processo proposto é dividido em três partes: segmentação textual, em que um determinado texto é dividido em pequenas frases ou palavras; extração de características, em que o resultado da etapa anterior é mapeado com 3 dicionários de sentimentos para descobrir os possíveis sentimentos de uma palavra, assim como a força e a polarização de tal sentimento; e, por último, a classificação multirrótulo. O estudo concluiu que os algoritmos RAkEl e ECC foram os melhores na base HR, enquanto que CLR, HOMER e ECC mostraram melhores resultados na base AI.
